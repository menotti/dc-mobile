\usepackage{amssymb}                        %%% para simbolo de 'marca registrada'
\usepackage[brazil]{varioref}               %%% referencias com página \vref
\usepackage[sf,bf,compact,topmarks,calcwidth,pagestyles]{titlesec} %%% definir títulos de seção
\usepackage{amsmath,amsfonts,amstext,amsthm,textcomp}
\usepackage{fancybox}                       %%% boxes
\usepackage[dvipsnames,usenames]{color}                          %%% Cores de fonte e fundo
\usepackage[dvipsnames,usenames]{xcolor}
\usepackage{colortbl}                       %%% Cores em tabelas
\usepackage{rotating}
\usepackage{fancyvrb}                       %%% Inclusão de texto usando VerbatimInput
\usepackage{bookman}                        %%% Fonte de Letras
\usepackage{enumerate}
\usepackage{lettrine}

	
\usepackage{framed}
\usepackage{enumitem}
\usepackage[htt]{hyphenat}

\usepackage{hyperref}
\hypersetup{
%  backref, %omitir na versão final
  pdfsubject =	{...},
  pdftitle =	{...},
  pdfkeywords = {...},
  pdfauthor =	{Ricardo Menotti, Daniel Lucrédio, ...},
  colorlinks =	{true}, %true na versão final
  linkcolor =	{black!50!blue},
  citecolor =	{black!50!blue},
  urlcolor =	{black!50!blue},
}

\usepackage[chapter]{minted}
\usepackage{caption}

%\usepackage{listings}
%\lstset{
%  numbers=left,
%  numberstyle=\tiny,
%  mathescape=true,
%  frame=single,
%  stepnumber=1,
%%  basicstyle=\scriptsize, % fontes menores nos codigos
%  keywordstyle=\ttfamily,
%  identifierstyle=\ttfamily, % \bfseries negrito nos codigos
%  commentstyle=\ttfamily
%}
\usepackage{pstricks} % listings color: black!50!red

\usepackage{listings}
\lstset{
	numbers=left,
	stepnumber=1,
	firstnumber=1,
	numberstyle=\tiny,
	mathescape=true,
	extendedchars=true,
	breaklines=true,
	frame=single,
	basicstyle=\footnotesize,
%	basicstyle=\scriptsize,
	stringstyle=\ttfamily,
% 	moredelim=*[l][\itshape]{||},%
	showstringspaces=false,
	float=h,
}
\lstdefinelanguage{lalp} {
  sensitive=true,
  keywordstyle={\color{black!50!green}\ttfamily\bfseries},
  keywords={const, typedef, fixed, in, out, counter, when, block_ram, delay_op, mult_op_s, add_reg_op_s},
  otherkeywords={<-},
  otherkeywords={@},
  commentstyle={\color{black!50!red}\itshape},
  morecomment=[l]{//}, 
  morecomment=[s]{/*}{*/},
}
\lstdefinelanguage{ican} {
  sensitive=true,
  keywordstyle={\color{black!50!red}\ttfamily\bfseries},
  keywords={array, begin, boolean, by, case, character, default, do, each, elif, else, end, enum, esac, false, fi, for, goto, if, in, inout, integer, nil, od, of, out, procedure, real, record, repeat, return, returns, sequence, set, to, true, until, where, while},
  commentstyle={\color{black!50!green}\itshape},
  morecomment=[l]{||},
  literate={{(X)}{$\times$}1 {(E)}{$\in$}1 {(U)}{$\cup$}1 {(N)}{$\cap$}1 {(<)}{$\langle$}1 {(>)}{$\rangle$}1 {->}{$\rightarrow$}2 {<\-}{$\leftarrow$}2 {=}{$=$}1 {>}{$>$}1 {<}{$<$}1 {!=}{$\neq$}1 {>=}{$\ge$}1 {<=}{$\le$}1}
}

\renewcommand\listingscaption{Algoritmo}
\renewcommand\listoflistingscaption{Lista de Algoritmos}

\renewcommand{\listoflistings}{%
  \addcontentsline{toc}{chapter}{\listoflistingscaption}%
  \listof{listing}{\listoflistingscaption}%
}

\usepackage[english,brazil]{babel}
\usepackage[utf8]{inputenc}
\usepackage[T1]{fontenc}
\usepackage{times}
\usepackage{graphicx,xr}
\usepackage{subfigure}
\usepackage{longtable}
\usepackage{setspace}
\usepackage{multirow}
\usepackage{rotating}
\usepackage{indentfirst}
\usepackage{xspace}
\usepackage[sort]{natbib}
\usepackage[a4paper,top=30mm,bottom=20mm,left=30mm,right=20mm]{geometry}


\exhyphenpenalty = 10000

\onehalfspace

\hyphenation{es-ta-be-le-ci-das a-tu-al-men-te Simple-Scalar-ARM}

\newcommand{\up}[1]{\raisebox{1.5ex}[0pt]{#1}}

\newcommand{\bi}{\begin{itemize}}
\newcommand{\ei}{\end{itemize}}
\newcommand{\be}{\begin{enumerate}}
\newcommand{\ee}{\end{enumerate}}

\definecolor{Gray}{gray}{0.9}

% using \autoref{} instead
%\newcommand{\reffig}[1]{Figura~\ref{fig:#1}}
%\newcommand{\reftab}[1]{Tabela~\ref{tab:#1}}
%\newcommand{\refcha}[1]{Capítulo~\ref{cha:#1}}
%\newcommand{\refses}[1]{Seção~\ref{ses:#1}}
%\newcommand{\refsub}[1]{Subseção~\ref{sub:#1}}
%\newcommand{\refape}[1]{Apêndice~\ref{ape:#1}}
%\newcommand{\refcod}[1]{Código~\ref{cod:#1}}
\newcommand{\subfigureautorefname}{\figureautorefname}

\newcommand{\cha}[2]{\chapter{#2}\label{cha:#1}} %\thispagestyle{empty}
\newcommand{\ses}[2]{\section{#2}\label{ses:#1}}
\newcommand{\sub}[2]{\subsection{#2}\label{sub:#1}}
\newcommand{\ape}[2]{\chapter{#2}\label{ape:#1}} %\thispagestyle{empty}

\newcommand{\figsim}[2]{
\begin{figure}[!ht]
  \centering
  \includegraphics[width=\textwidth]{../figuras/#1.jpg}
  \caption{#2}
  \label{fig:#1}
\end{figure}
}

%%DICA
\newcommand{\dica}[1]{
\begin{framed}
\textbf{Dica:} #1
\end{framed}
}

\newcommand{\ew}[1]{\selectlanguage{english}\emph{#1}\selectlanguage{brazil}}
%\newcommand{\ew}[1]{\emph{#1}}

\renewcommand{\bibname}{Referências Bibliográficas}

\setcounter{secnumdepth}{2}

\setcounter{tocdepth}{2}

\pagestyle{empty}

%\font\numberfont= goxi2074 scaled 2000      %%% Fonte para o Número do Capítulo
\font\numberfont= pzcmi scaled 6500      %%% Fonte para o Número do Capítulo

                                            %%% redefine o formato do título
\titleformat{\chapter}[display]
  {\normalfont\Large\sffamily
  }
  {
   \rule[32pt]{.7\linewidth}{4pt}
   \hspace{-10pt}
   \shadowbox{
   \begin{minipage}{.18\linewidth}
     \begin{center}
       \textsc{\Large\chaptertitlename}\\
       \vspace{1ex}
       {\numberfont \thechapter}\\
       \vspace{1ex}
     \end{center}
   \end{minipage}}
  }
  {0pt}
  {\filcenter
   \Huge
   }
  [\hfill\rule{.8\textwidth}{0.75pt}\\
     \vskip-1.8ex\hfill\rule{.7\textwidth}{2pt}]


\newpagestyle{body}{ %[\small\sffamily]{
\headrule

\sethead[\thepage][][\ifthechapter{\thechapter\quad}{} \textsl{\chaptertitle}]%
          {\ifthechapter{\thechapter\quad \textsl{\chaptertitle}}{\textsl{\chaptertitle}}}{}{\thepage}


}

\newpagestyle{misc}{
  \headrule
  \sethead{\textsl{\chaptertitle}}{}{\thepage}
  \setfoot{}{}{}
}

\font\largefont= pzcmi scaled 6500

\newcommand{\versal}[1]{{\noindent
    \setbox0\hbox{\largefont #1 }%
    \count0=\ht0                   % height of versal
    \count1=\baselineskip          % baselineskip
    \divide\count0 by \count1      % versal height/baselineskip
    \dimen1 = \count0\baselineskip % distance to drop versal
    \advance\count0 by 1\relax     % no of indented lines
    \dimen0=\wd0                   % width of versal
    \global\hangindent\dimen0      % set indentation distance
    \global\hangafter-\count0      % set no of indented lines
    \hskip-\dimen0\setbox0\hbox to\dimen0{\raise-\dimen1\box0\hss}%
    \dp0=0in\ht0=0in\box0}}


%%%  define linha mais grossa para tabelas

\newdimen\arrayruleHwidth
\setlength{\arrayruleHwidth}{2pt} \makeatletter
\def\Hline{\noalign{\ifnum0=`}\fi\hrule \@height \arrayruleHwidth
\futurelet \@tempa\@xhline} \makeatother

\definecolor{mygreen}{rgb}{0.1, 0.6, 0.2}
