\documentclass[a4paper,12pt,brazil,doubleside]{book}

% tentei incluir o da pasta comum mas não funcionou
% \usepackage{amssymb}                        %%% para simbolo de 'marca registrada'
\usepackage[brazil]{varioref}               %%% referencias com página \vref
\usepackage[sf,bf,compact,topmarks,calcwidth,pagestyles]{titlesec} %%% definir títulos de seção
\usepackage{amsmath,amsfonts,amstext,amsthm,textcomp}
\usepackage{fancybox}                       %%% boxes
\usepackage[dvipsnames,usenames]{color}                          %%% Cores de fonte e fundo
\usepackage[dvipsnames,usenames]{xcolor}
\usepackage{colortbl}                       %%% Cores em tabelas
\usepackage{rotating}
\usepackage{fancyvrb}                       %%% Inclusão de texto usando VerbatimInput
\usepackage{bookman}                        %%% Fonte de Letras
\usepackage{enumerate}
\usepackage{lettrine}
\usepackage{minted}	
\usepackage{caption}
\usepackage{framed}
\usepackage{enumitem}
\usepackage[htt]{hyphenat}


%\usepackage{listings}
%\lstset{
%  numbers=left,
%  numberstyle=\tiny,
%  mathescape=true,
%  frame=single,
%  stepnumber=1,
%%  basicstyle=\scriptsize, % fontes menores nos codigos
%  keywordstyle=\ttfamily,
%  identifierstyle=\ttfamily, % \bfseries negrito nos codigos
%  commentstyle=\ttfamily
%}
\usepackage{pstricks} % listings color: black!50!red

\usepackage{listings}
\lstset{
	numbers=left,
	stepnumber=1,
	firstnumber=1,
	numberstyle=\tiny,
	mathescape=true,
	extendedchars=true,
	breaklines=true,
	frame=single,
	basicstyle=\footnotesize,
%	basicstyle=\scriptsize,
	stringstyle=\ttfamily,
% 	moredelim=*[l][\itshape]{||},%
	showstringspaces=false,
	float=h,
}
\lstdefinelanguage{lalp} {
  sensitive=true,
  keywordstyle={\color{black!50!green}\ttfamily\bfseries},
  keywords={const, typedef, fixed, in, out, counter, when, block_ram, delay_op, mult_op_s, add_reg_op_s},
  otherkeywords={<-},
  otherkeywords={@},
  commentstyle={\color{black!50!red}\itshape},
  morecomment=[l]{//}, 
  morecomment=[s]{/*}{*/},
}
\lstdefinelanguage{ican} {
  sensitive=true,
  keywordstyle={\color{black!50!red}\ttfamily\bfseries},
  keywords={array, begin, boolean, by, case, character, default, do, each, elif, else, end, enum, esac, false, fi, for, goto, if, in, inout, integer, nil, od, of, out, procedure, real, record, repeat, return, returns, sequence, set, to, true, until, where, while},
  commentstyle={\color{black!50!green}\itshape},
  morecomment=[l]{||},
  literate={{(X)}{$\times$}1 {(E)}{$\in$}1 {(U)}{$\cup$}1 {(N)}{$\cap$}1 {(<)}{$\langle$}1 {(>)}{$\rangle$}1 {->}{$\rightarrow$}2 {<\-}{$\leftarrow$}2 {=}{$=$}1 {>}{$>$}1 {<}{$<$}1 {!=}{$\neq$}1 {>=}{$\ge$}1 {<=}{$\le$}1}
}

\renewcommand\listingscaption{Algoritmo}
\renewcommand\listoflistingscaption{Lista de Algoritmos}


\usepackage[english,brazil]{babel}
\usepackage[utf8]{inputenc}
\usepackage[T1]{fontenc}
\usepackage{times}
\usepackage{graphicx,xr}
\usepackage{subfigure}
\usepackage{longtable}
\usepackage{setspace}
\usepackage{multirow}
\usepackage{rotating}
\usepackage{indentfirst}
\usepackage{xspace}
\usepackage[sort]{natbib}
\usepackage[a4paper,top=30mm,bottom=20mm,left=30mm,right=20mm]{geometry}

\usepackage{hyperref}
\hypersetup{
%  backref, %omitir na versão final
  pdfsubject =	{...},
  pdftitle =	{...},
  pdfkeywords = {...},
  pdfauthor =	{Ricardo Menotti, Daniel Lucrédio, ...},
  colorlinks =	{true}, %true na versão final
  linkcolor =	{black!50!blue},
  citecolor =	{black!50!blue},
  urlcolor =	{black!50!blue},
}

\exhyphenpenalty = 10000

\onehalfspace

\hyphenation{es-ta-be-le-ci-das a-tu-al-men-te Simple-Scalar-ARM}

\newcommand{\up}[1]{\raisebox{1.5ex}[0pt]{#1}}

\newcommand{\bi}{\begin{itemize}}
\newcommand{\ei}{\end{itemize}}
\newcommand{\be}{\begin{enumerate}}
\newcommand{\ee}{\end{enumerate}}

\definecolor{Gray}{gray}{0.9}

% using \autoref{} instead
%\newcommand{\reffig}[1]{Figura~\ref{fig:#1}}
%\newcommand{\reftab}[1]{Tabela~\ref{tab:#1}}
%\newcommand{\refcha}[1]{Capítulo~\ref{cha:#1}}
%\newcommand{\refses}[1]{Seção~\ref{ses:#1}}
%\newcommand{\refsub}[1]{Subseção~\ref{sub:#1}}
%\newcommand{\refape}[1]{Apêndice~\ref{ape:#1}}
%\newcommand{\refcod}[1]{Código~\ref{cod:#1}}
\newcommand{\subfigureautorefname}{\figureautorefname}

\newcommand{\cha}[2]{\chapter{#2}\label{cha:#1}} %\thispagestyle{empty}
\newcommand{\ses}[2]{\section{#2}\label{ses:#1}}
\newcommand{\sub}[2]{\subsection{#2}\label{sub:#1}}
\newcommand{\ape}[2]{\chapter{#2}\label{ape:#1}} %\thispagestyle{empty}

\newcommand{\figsim}[2]{
\begin{figure}[!ht]
  \centering
  \includegraphics[width=\textwidth]{../figuras/#1.jpg}
  \caption{#2}
  \label{fig:#1}
\end{figure}
}

\newcommand{\ew}[1]{\selectlanguage{english}\emph{#1}\selectlanguage{brazil}}
%\newcommand{\ew}[1]{\emph{#1}}

\renewcommand{\bibname}{Referências Bibliográficas}

\setcounter{secnumdepth}{2}

\setcounter{tocdepth}{2}

\pagestyle{empty}

%\font\numberfont= goxi2074 scaled 2000      %%% Fonte para o Número do Capítulo
\font\numberfont= pzcmi scaled 6500      %%% Fonte para o Número do Capítulo

                                            %%% redefine o formato do título
\titleformat{\chapter}[display]
  {\normalfont\Large\sffamily
  }
  {
   \rule[32pt]{.7\linewidth}{4pt}
   \hspace{-10pt}
   \shadowbox{
   \begin{minipage}{.18\linewidth}
     \begin{center}
       \textsc{\Large\chaptertitlename}\\
       \vspace{1ex}
       {\numberfont \thechapter}\\
       \vspace{1ex}
     \end{center}
   \end{minipage}}
  }
  {0pt}
  {\filcenter
   \Huge
   }
  [\hfill\rule{.8\textwidth}{0.75pt}\\
     \vskip-1.8ex\hfill\rule{.7\textwidth}{2pt}]


\newpagestyle{body}{ %[\small\sffamily]{
\headrule

\sethead[\thepage][][\ifthechapter{\thechapter\quad}{} \textsl{\chaptertitle}]%
          {\ifthechapter{\thechapter\quad \textsl{\chaptertitle}}{\textsl{\chaptertitle}}}{}{\thepage}


}

\newpagestyle{misc}{
  \headrule
  \sethead{\textsl{\chaptertitle}}{}{\thepage}
  \setfoot{}{}{}
}

\font\largefont= pzcmi scaled 6500

\newcommand{\versal}[1]{{\noindent
    \setbox0\hbox{\largefont #1 }%
    \count0=\ht0                   % height of versal
    \count1=\baselineskip          % baselineskip
    \divide\count0 by \count1      % versal height/baselineskip
    \dimen1 = \count0\baselineskip % distance to drop versal
    \advance\count0 by 1\relax     % no of indented lines
    \dimen0=\wd0                   % width of versal
    \global\hangindent\dimen0      % set indentation distance
    \global\hangafter-\count0      % set no of indented lines
    \hskip-\dimen0\setbox0\hbox to\dimen0{\raise-\dimen1\box0\hss}%
    \dp0=0in\ht0=0in\box0}}


%%%  define linha mais grossa para tabelas

\newdimen\arrayruleHwidth
\setlength{\arrayruleHwidth}{2pt} \makeatletter
\def\Hline{\noalign{\ifnum0=`}\fi\hrule \@height \arrayruleHwidth
\futurelet \@tempa\@xhline} \makeatother

\definecolor{mygreen}{rgb}{0.1, 0.6, 0.2}

\usepackage{amssymb}                        %%% para simbolo de 'marca registrada'
\usepackage[brazil]{varioref}               %%% referencias com página \vref
\usepackage[sf,bf,compact,topmarks,calcwidth,pagestyles]{titlesec} %%% definir títulos de seção
\usepackage{amsmath,amsfonts,amstext,amsthm,textcomp}
\usepackage{fancybox}                       %%% boxes
\usepackage[dvipsnames,usenames]{color}                          %%% Cores de fonte e fundo
\usepackage[dvipsnames,usenames]{xcolor}
\usepackage{colortbl}                       %%% Cores em tabelas
\usepackage{rotating}
\usepackage{fancyvrb}                       %%% Inclusão de texto usando VerbatimInput
\usepackage{bookman}                        %%% Fonte de Letras
\usepackage{enumerate}
\usepackage{lettrine}
\usepackage{minted}	
\usepackage{caption}
\usepackage{framed}
\usepackage{enumitem}
\usepackage[htt]{hyphenat}


%\usepackage{listings}
%\lstset{
%  numbers=left,
%  numberstyle=\tiny,
%  mathescape=true,
%  frame=single,
%  stepnumber=1,
%%  basicstyle=\scriptsize, % fontes menores nos codigos
%  keywordstyle=\ttfamily,
%  identifierstyle=\ttfamily, % \bfseries negrito nos codigos
%  commentstyle=\ttfamily
%}
\usepackage{pstricks} % listings color: black!50!red

\usepackage{listings}
\lstset{
	numbers=left,
	stepnumber=1,
	firstnumber=1,
	numberstyle=\tiny,
	mathescape=true,
	extendedchars=true,
	breaklines=true,
	frame=single,
	basicstyle=\footnotesize,
%	basicstyle=\scriptsize,
	stringstyle=\ttfamily,
% 	moredelim=*[l][\itshape]{||},%
	showstringspaces=false,
	float=h,
}
\lstdefinelanguage{lalp} {
  sensitive=true,
  keywordstyle={\color{black!50!green}\ttfamily\bfseries},
  keywords={const, typedef, fixed, in, out, counter, when, block_ram, delay_op, mult_op_s, add_reg_op_s},
  otherkeywords={<-},
  otherkeywords={@},
  commentstyle={\color{black!50!red}\itshape},
  morecomment=[l]{//}, 
  morecomment=[s]{/*}{*/},
}
\lstdefinelanguage{ican} {
  sensitive=true,
  keywordstyle={\color{black!50!red}\ttfamily\bfseries},
  keywords={array, begin, boolean, by, case, character, default, do, each, elif, else, end, enum, esac, false, fi, for, goto, if, in, inout, integer, nil, od, of, out, procedure, real, record, repeat, return, returns, sequence, set, to, true, until, where, while},
  commentstyle={\color{black!50!green}\itshape},
  morecomment=[l]{||},
  literate={{(X)}{$\times$}1 {(E)}{$\in$}1 {(U)}{$\cup$}1 {(N)}{$\cap$}1 {(<)}{$\langle$}1 {(>)}{$\rangle$}1 {->}{$\rightarrow$}2 {<\-}{$\leftarrow$}2 {=}{$=$}1 {>}{$>$}1 {<}{$<$}1 {!=}{$\neq$}1 {>=}{$\ge$}1 {<=}{$\le$}1}
}

\renewcommand\listingscaption{Algoritmo}
\renewcommand\listoflistingscaption{Lista de Algoritmos}


\usepackage[english,brazil]{babel}
\usepackage[utf8]{inputenc}
\usepackage[T1]{fontenc}
\usepackage{times}
\usepackage{graphicx,xr}
\usepackage{subfigure}
\usepackage{longtable}
\usepackage{setspace}
\usepackage{multirow}
\usepackage{rotating}
\usepackage{indentfirst}
\usepackage{xspace}
\usepackage[sort]{natbib}
\usepackage[a4paper,top=30mm,bottom=20mm,left=30mm,right=20mm]{geometry}

\usepackage{hyperref}
\hypersetup{
%  backref, %omitir na versão final
  pdfsubject =	{...},
  pdftitle =	{...},
  pdfkeywords = {...},
  pdfauthor =	{Ricardo Menotti, Daniel Lucrédio, ...},
  colorlinks =	{true}, %true na versão final
  linkcolor =	{black!50!blue},
  citecolor =	{black!50!blue},
  urlcolor =	{black!50!blue},
}

\exhyphenpenalty = 10000

\onehalfspace

\hyphenation{es-ta-be-le-ci-das a-tu-al-men-te Simple-Scalar-ARM}

\newcommand{\up}[1]{\raisebox{1.5ex}[0pt]{#1}}

\newcommand{\bi}{\begin{itemize}}
\newcommand{\ei}{\end{itemize}}
\newcommand{\be}{\begin{enumerate}}
\newcommand{\ee}{\end{enumerate}}

\definecolor{Gray}{gray}{0.9}

% using \autoref{} instead
%\newcommand{\reffig}[1]{Figura~\ref{fig:#1}}
%\newcommand{\reftab}[1]{Tabela~\ref{tab:#1}}
%\newcommand{\refcha}[1]{Capítulo~\ref{cha:#1}}
%\newcommand{\refses}[1]{Seção~\ref{ses:#1}}
%\newcommand{\refsub}[1]{Subseção~\ref{sub:#1}}
%\newcommand{\refape}[1]{Apêndice~\ref{ape:#1}}
%\newcommand{\refcod}[1]{Código~\ref{cod:#1}}
\newcommand{\subfigureautorefname}{\figureautorefname}

\newcommand{\cha}[2]{\chapter{#2}\label{cha:#1}} %\thispagestyle{empty}
\newcommand{\ses}[2]{\section{#2}\label{ses:#1}}
\newcommand{\sub}[2]{\subsection{#2}\label{sub:#1}}
\newcommand{\ape}[2]{\chapter{#2}\label{ape:#1}} %\thispagestyle{empty}

\newcommand{\figsim}[2]{
\begin{figure}[!ht]
  \centering
  \includegraphics[width=\textwidth]{../figuras/#1.jpg}
  \caption{#2}
  \label{fig:#1}
\end{figure}
}

\newcommand{\ew}[1]{\selectlanguage{english}\emph{#1}\selectlanguage{brazil}}
%\newcommand{\ew}[1]{\emph{#1}}

\renewcommand{\bibname}{Referências Bibliográficas}

\setcounter{secnumdepth}{2}

\setcounter{tocdepth}{2}

\pagestyle{empty}

%\font\numberfont= goxi2074 scaled 2000      %%% Fonte para o Número do Capítulo
\font\numberfont= pzcmi scaled 6500      %%% Fonte para o Número do Capítulo

                                            %%% redefine o formato do título
\titleformat{\chapter}[display]
  {\normalfont\Large\sffamily
  }
  {
   \rule[32pt]{.7\linewidth}{4pt}
   \hspace{-10pt}
   \shadowbox{
   \begin{minipage}{.18\linewidth}
     \begin{center}
       \textsc{\Large\chaptertitlename}\\
       \vspace{1ex}
       {\numberfont \thechapter}\\
       \vspace{1ex}
     \end{center}
   \end{minipage}}
  }
  {0pt}
  {\filcenter
   \Huge
   }
  [\hfill\rule{.8\textwidth}{0.75pt}\\
     \vskip-1.8ex\hfill\rule{.7\textwidth}{2pt}]


\newpagestyle{body}{ %[\small\sffamily]{
\headrule

\sethead[\thepage][][\ifthechapter{\thechapter\quad}{} \textsl{\chaptertitle}]%
          {\ifthechapter{\thechapter\quad \textsl{\chaptertitle}}{\textsl{\chaptertitle}}}{}{\thepage}


}

\newpagestyle{misc}{
  \headrule
  \sethead{\textsl{\chaptertitle}}{}{\thepage}
  \setfoot{}{}{}
}

\font\largefont= pzcmi scaled 6500

\newcommand{\versal}[1]{{\noindent
    \setbox0\hbox{\largefont #1 }%
    \count0=\ht0                   % height of versal
    \count1=\baselineskip          % baselineskip
    \divide\count0 by \count1      % versal height/baselineskip
    \dimen1 = \count0\baselineskip % distance to drop versal
    \advance\count0 by 1\relax     % no of indented lines
    \dimen0=\wd0                   % width of versal
    \global\hangindent\dimen0      % set indentation distance
    \global\hangafter-\count0      % set no of indented lines
    \hskip-\dimen0\setbox0\hbox to\dimen0{\raise-\dimen1\box0\hss}%
    \dp0=0in\ht0=0in\box0}}


%%%  define linha mais grossa para tabelas

\newdimen\arrayruleHwidth
\setlength{\arrayruleHwidth}{2pt} \makeatletter
\def\Hline{\noalign{\ifnum0=`}\fi\hrule \@height \arrayruleHwidth
\futurelet \@tempa\@xhline} \makeatother

\definecolor{mygreen}{rgb}{0.1, 0.6, 0.2}


\title{Título}

\begin{document}
\selectlanguage{brazil}


\pagestyle{empty}

\cleardoublepage

\selectlanguage{brazil}
\onehalfspace

\pagestyle{plain}
\pagenumbering{roman}

\setcounter{tocdepth}{1} % 0 capítulos, 1 seções, 2 subseções
\tableofcontents
\clearpage % se o verso ficar em branco...% VAI
\thispagestyle{empty}

\listoffigures
\addcontentsline{toc}{section}{Lista de Figuras}
\clearpage % se o verso ficar em branco...% VAI
\thispagestyle{empty}

\listoftables
\addcontentsline{toc}{section}{Lista de Tabelas}
\clearpage % se o verso ficar em branco...% VAI
\thispagestyle{empty}

\lstlistoflistings
\addcontentsline{toc}{section}{Lista de Algoritmos}
\clearpage % se o verso ficar em branco...% VAI
\thispagestyle{empty}

\chapter*{Lista de Abreviaturas}
\addcontentsline{toc}{section}{Lista de Abreviaturas}
\begin{longtable}{ll}
\end{longtable}

\renewcommand\lstlistingname{Algorithm}
\renewcommand\lstlistlistingname{Algorithms}

\chapter*{Resumo}
\addcontentsline{toc}{section}{Resumo}

\begin{singlespace}
Esse material didático oferece ao aluno de graduação uma visão geral de como programar para o sistema móvel Android e utilizar suas APIs nativas na criação de aplicativos. O material tentará cobrir desde o básico como a configuração do ambiente de desenvolvimento, criação de layouts básicos e complexos, estrutura geral de um aplicativo e ir até a programação de aplicativos mais complexos que tentam utilizar uma ou várias APIs em conjunto.

O objetivo é dar apenas uma noção de como utilizar as ferramentas do Android, introduzir ao aluno os conceitos e não entrar em detalhes do sistema operacional ou em conceitos mais aprofundados, mas sim uma visão genérica.
Após a leitura desse material e realização da prática o aluno deve estar preparado para construir seus próprios aplicativos nativos, e poderá até monetizar seus aplicativos se desejar.
\end{singlespace}

\clearpage % se o verso ficar em branco...
\thispagestyle{empty}

\thispagestyle{empty}

\doublespace
%\onehalfspace
%\singlespace

\cleardoublepage

\pagestyle{body}
\pagenumbering{arabic}

%%%%%%%%%%%%%%%%%%%%%%%%%%%%%%%%%%%%%%%%%%%%%%%%%%%%%%%%%%%%%%%%%%%%%%%%%%%%%%%%
\cha{introducao}{Introdução}
%\setcounter{page}{1}
%7-10p

\begin{figure}[!htb]
  \centering
  \includegraphics[width=.5\textwidth]{../figuras/LogoDC.jpg}
  \caption{Departamento de Computação}
  \label{fig:dc}
\end{figure}

**** \emph{Não sei o que escrever na introdução que eu não tenha escrito no resumo} ****
% http://tug.ctan.org/tex-archive/macros/latex/contrib/minted/
% http://pygments.org/

\cha{configambiente}{Configuração do Ambiente}
A instalação e configuração do ambiente de desenvolvimento para Android é simples, o Google fornece um pacote chamado ADT (Android Development Tools) que contém o ambiente Eclipse com o plugin do Android, algumas ferramentas para instalação dos aplicativos nos smartphones, o gerenciador do SDK e as imagens para o emulador do Android. Essas ferramentas são suficientes para o desenvolvimento na plataforma.
O pacote ADT pode ser encontrado em: \href{http://developer.android.com/sdk/}{Android SDK}\footnote{\href{http://developer.android.com/sdk/}{http://developer.android.com/sdk/}}. 

Basta fazer o download do pacote e extrair que tudo já está pré-configurado para iniciar o desenvolvimento, portanto não há muito o que configurar.

\cha{intro-android}{Linguagem do Android}
\section{Linguagem}
A linguagem usada para programar na plataforma Android é baseada em Java. Então antes de engajar no aprendizado Android é altamente recomendável estudar material Java e principalmente o paradigma de orientação à objetos.

A diferença entre o Java convencional e o Android então se dá na organização e configuração feita através de arquivos XML especificos do Android. Alguns XML servem para configurar o aplicativo, layout de cada tela, outro dá suporte a strings para facilitar o suporte a multiplos idiomas.
Felizmente o conjunto Eclipse com ADT já cuida disso automaticamente ou então facilita através de interfaces gráficas para os programadores, então para qualquer iniciante nessa área e recomendável a utilizacão do ambiente Eclipse.

A criação de layouts dos aplicativos pode ser feita inteiramente através da interface gráfica disponível no ambiente, no melhor estilo \textit{drag and drop}. 

\section{Entendendo os arquivos XML}
Dentre os diversos arquivos XML existentes na configuração de um aplicativo Android o mais importante é o \texttt{\textcolor{mygreen}{AndroidManifest.xml}} pois é nele que se exprime as configurações gerais do aplicativo, não iremos adentrar muito nos detalhes das configurações, mas apenas deixar claro que é nesse arquivo que se coloca as versões do Android que seu aplicativo será compatível com, as permissões para usar os recursos do aparelho como Internet, GPS, Bluetooth, etc. 

Dentro da pasta \texttt{\textcolor{mygreen}{res/}} de recursos, encontram-se outros arquivos, estes que são referentes a disposição do layout e a valores de strings. A pasta \texttt{\textcolor{mygreen}{layout/}} junto com as pastas \texttt{\textcolor{mygreen}{drawable-*/}} são para dispor o layout, cada \textit{drawable} é para comportar artefatos para cada tamanho diferente de tela, enquanto que a \textit{layout} é a dispoção geral do layout, nesse arquivo é que se coloca os itens que irão nas telas, como botões, caixas de texto, caixas de seleção, etc.

Na pasta \texttt{\textcolor{mygreen}{values/}} o mais importante é o arquivo \texttt{\textcolor{mygreen}{strings.xml}} que se escreve os valores das strings do aplicativo, sempre que quiser referenciar alguma string, essa deverá estar expressa nesse arquivo. Fica fácil dessa forma fazer o aplicativo suportar múltiplos idiomas através da tradução desse conteúdo para outros idiomas. 

Tudo isso será trabalhado a partir de exemplos mais pra frente no material.


\cha{primeiro-aplicativo}{Criando seu primeiro aplicativo}
Para exemplificar a criação de um aplicativo, seguiremos o exemplo dado pelo próprio manual do Google sobre o Android (Ver original\footnote{Original em: \href{http://developer.android.com/training/basics/firstapp/creating-project.html}{http://developer.android.com/training/basics/firstapp/creating-project.html}}). Um aplicativo simples do tipo \textit{''Hello World''}.

Iniciaremos criando um novo projeto no Eclipse: \textit{File -> New -> Android Application Project}.

Na janela que apareceu você deve colocar o nome do aplicativo, do projeto e do pacote. O nome do pacote deve seguir a convenção do Java\footnote{Convenção sobre nome dos pacotes: \href{http://docs.oracle.com/javase/tutorial/java/package/namingpkgs.html}{http://docs.oracle.com/javase/tutorial/java/package/namingpkgs.html}}.
\begin{itemize}
	\item \textit{Minimum Required SDK:} É a versão mínima do sistema operacional Android que sua aplicação irá suportar, o mais comum é a versão 8 do SDK que se refere ao Android 2.2, alguns tipos de layouts mais complexos não são suportados em versões mais antigas
	\item \textit{Target SDK:} É a versão principal para qual seu aplicativo está sendo desenvolvido
	\item \textit{Compile With:} Versão para qual seu aplicativo será compilado
	\item \textit{Theme:} Cores do layout
\end{itemize}


\begin{figure}[H]
  \centering
  \includegraphics[width=.8\textwidth]{figuras/1-criando-app.jpg}
  \caption{Primeira janela de criação de novo aplicativo}
  \label{fig:d}
  \includegraphics[width=.8\textwidth]{figuras/2-criando-app.jpg}
  \caption{Segunda janela de criação de novo aplicativo}
  \label{fig:c}
\end{figure}


\begin{singlespace}
\begin{listing}[H]
\inputminted[linenos=true,fontsize=\small,frame=lines, framesep=2mm, tabsize=2,numbersep=5pt]{xml}{../../src/android/firstapp/sdk-manifest.xml}
\label{AndroidManifest.sdk}
\caption{Exemplo de configuração de versão do SDK no \texttt{\textcolor{mygreen}{AndroidManifest.xml}} }
\end{listing}

Primeiro vamos criar um layout para o aplicativo usando a interface gráfica do Eclipse, primeiro abra o arquivo \texttt{\textcolor{mygreen}{res/layout/activity\_main.xml}} , segundo o AndroidManifest, é essa \texttt{activity} que será aberta quando o aplicativo for iniciado, configurado através do \textit{intent-filter}. Cada \texttt{activity} é uma tela do aplicativo e é responsáve primariamente por mostrar as informações, o Android segue o modelo \emph{MVC} \textit{(Model-View-Control)}, e as \texttt{activity} são as \textit{Views} do aplicativo.
\begin{listing}
\begin{minted}[mathescape,
               linenos,
               numbersep=5pt,
               tabsize=2,
               frame=lines,
               framesep=2mm]{xml}
<intent-filter>
	<action android:name="android.intent.action.MAIN" />
	<category android:name="android.intent.category.LAUNCHER"/>
</intent-filter>
\end{minted}
\caption{Configuração dos \textit{intent-filters} no \texttt{\textcolor{mygreen}{AndroidManifest.xml}} }
\end{listing}

Não entraremos em detalhes sobre os \textit{intent-filters} agora. Selecione o ''Hello world'' e delete ele da sua \texttt{activity}

\begin{figure}[H]
  \centering
  \includegraphics{figuras/3-criando-app.png}
  \caption{Selecionando o Hello world}
  \label{fig:b}
\end{figure}

Agora arraste um \textit{Text Field -> Plain Text} e um \textit{Form Widgets -> Button} para sua \texttt{activity} 

\begin{figure}[H]
  \centering
  \includegraphics{figuras/4-criando-app.png}
  \caption{\texttt{activity} com os elementos colocados na tela}
  \label{fig:a}
\end{figure}

Ao clicar duas vezes no elemento no modo visual, você sera levado ao correspondente marcador desse elemento no XML correspondente da \texttt{activity}. Clique duas vezes na caixa de texto.

\begin{listing}
\inputminted[linenos=true,fontsize=\small,frame=lines, framesep=2mm, tabsize=2,numbersep=5pt]{xml}{../../src/android/firstapp/1.xml}
\caption{Código da caixa de texto}
\end{listing}

Primeiro, modifique a \textit{id} para um mais intuitivo, por exemplo você pode chama-lo de \textit{caixaTextoNome}. Todo \textit{id} deve ser precedido de \texttt{@+id/} como definido pelo Android. Os outros atributos são para definir o tamanho, alinhamento e margem, todos relativos ao elemento pai desse elemento que no caso é o próprio layout. Depois adicione uma \textit{hint} para essa caixa de texto, uma \textit{hint} é algo que vai estar escrito na caixa de texto quando ela estiver vazia, indicando que tipo de texto você pretende que seja escrito nessa caixa de texto, essa \textit{hint} é uma referência a \texttt{string} chamada \texttt{nome} que está definida no arquivo \texttt{\textcolor{mygreen}{res/values/strings.xml}}.	 No código acima a \textit{hint} já foi colocada na linha 10.

Depois modifique o código do botão, trocando o \textit{id} e as referências ao \textit{id} da caixa de texto, também edite o \texttt{android:text} para fazer uma referência a uma string no arquivo de strings. Por último adicione um atributo \texttt{android:onClick} que é o método que será executado quando o botão for pressionado.

\begin{listing}[H]
\inputminted[linenos=true,fontsize=\small,frame=lines, framesep=2mm, tabsize=2,numbersep=5pt]{xml}{../../src/android/firstapp/2.xml}
\caption{Código do botão}
\end{listing}

Adicione as strings ao arquivo de strings, essas strings podem ser referenciadas a qualquer momento tanto na construção do layout como na codificação do aplicativo.

\begin{listing}
\inputminted[linenos=true,fontsize=\small,frame=lines, framesep=2mm, tabsize=2,numbersep=5pt]{xml}{../../src/android/firstapp/3.xml}
\caption{Arquivo de strings com as duas strings adicionadas}
\end{listing}

Agora abra a classe \texttt{MainActivity} localizada na pasta \texttt{\textcolor{mygreen}{src/}} do seu projeto e adicione um novo método

\begin{listing}
\inputminted[linenos=true,fontsize=\small,frame=lines, framesep=2mm, tabsize=2,numbersep=5pt]{java}{../../src/android/firstapp/4.java}
\caption{Adicionando método à classe MainActivity}
\end{listing}

\bi
\item{Isso vai requer você importe a classe View, você pode apertar \texttt{Ctrl+Shit+O} no Eclipse para importar classes que estejam faltando}
\ei

\begin{listing}
\inputminted[linenos=true,fontsize=\small,frame=lines, framesep=2mm, tabsize=2,numbersep=5pt]{java}{../../src/android/firstapp/4-2.java}
\caption{Exemplo de import de classe Android}
\end{listing}


Primeiro, crie um novo \texttt{Intent}\footnote{Documentação Intent: \href{http://developer.android.com/reference/android/content/Intent.html}{http://developer.android.com/reference/android/content/Intent.html}}, um \texttt{Intent} é um objeto que providencia uma ligação entre dois componentes separados (tais como duas \textit{activities}). O \texttt{Intent} representa uma ''intenção do aplicativo de fazer algo'', eles podem ser usados para uma variedade de tarefas, mas são principalmente usados para iniciar uma nova \texttt{activity}.

\begin{listing}[H]
\inputminted[linenos=true,fontsize=\small,frame=lines, framesep=2mm, tabsize=2,numbersep=5pt]{java}{../../src/android/firstapp/5.java}
\caption{Adicionando uma \texttt{Intent}}
\end{listing}

Agora você precisa obter o texto que está escrito na caixa para fazer algo com ele, no caso iremos enviar para outra \texttt{activity} que irá mostrar esse texto.

\begin{listing}[H]
\inputminted[linenos=true,fontsize=\small,frame=lines, framesep=2mm, tabsize=2,numbersep=5pt]{java}{../../src/android/firstapp/6.java}
\caption{Obtendo o conteúdo da caixa de texto e enviando para outra \texttt{activity}}
\end{listing}

O código na linha 3 está obtendo a referência a caixa de texto usando o método \texttt{findViewById()} e o argumento passado é o \textit{id} da caixa de texto, observe que você chama com \texttt{R.id.caixaTextoNome} isso é feito dessa forma porque o android compila automaticamente uma classe R (R de \textit{Resources}) que contém as \textit{id's} e \textit{strings} criadas nos arquivos XML.
Em seguida usando o método \texttt{getText()} da caixa de texto, obtem-se a string que foi escrita lá.

Por fim, essa string é colocada no \texttt{Intent} com o método \texttt{putExtra()}, uma \texttt{Intent} pode carregar consigo uma coleção de vários tipos de dados como pares chave-valor chamados \textit{extras}, esse método toma a chave como primeiro parâmetro e o valor no segundo parâmetro.
Para que a próxima \texttt{activity} consiga coletar esse valor, você deve definir uma chave para seu \textit{extra} usando uma constante pública. Para isso adicione a definição de \texttt{EXTRA\_MESSAGE} no topo da sua classe \texttt{MainActivity}.

\begin{listing}[H]
\inputminted[linenos=true,fontsize=\small,frame=lines, framesep=2mm, tabsize=2,numbersep=5pt]{java}{../../src/android/firstapp/7.java}
\caption{Constante como chave para um extra}
\end{listing}

Agora você deve criar uma nova \texttt{activity}, para isso vá em \textit{File -> New -> Other -> Android Activity} e selecione \textit{Blank Activity}. Preencha a próxima janela dessa maneira, depois clique \textit{Finish}. 

\begin{figure}[H]
  \centering
  \includegraphics[width=1\textwidth]{figuras/5-criando-app.png}
  \caption{Criando uma nova \texttt{activity}}
  \label{fig:e}
\end{figure}

A classe já vem com alguns métodos implementados, alguns não serão necessários para esse aplicativo, mas mantenha eles na classe. Todas as classes que são subclasses de \texttt{Activity} precisam implementar o método \texttt{onCreate()}\footnote{\href{http://developer.android.com/reference/android/app/Activity.html\#onCreate(android.os.Bundle)}{http://developer.android.com/reference/android/app/Activity.html\#onCreate(android.os.Bundle)}}.

Geralmente você precisaria adicionar uma nova string com o nome da classe no XML, e adicionar a \texttt{activity} no \texttt{\textcolor{mygreen}{AndroidManifest.xml}} porém como estamos trabalhando com o Eclipse, você não precisa se preocupar que ele faz isso sozinho.

Agora, precisamos extrair os dados enviados a essa \texttt{activity} através do \texttt{intent}, você pode pegar o \texttt{intent} que começou a \texttt{activity} chamando o método \texttt{getIntent()}\footnote{\href{http://developer.android.com/reference/android/app/Activity.html\#getIntent()}{http://developer.android.com/reference/android/app/Activity.html\#getIntent()}}.

\begin{listing}[H]
\inputminted[linenos=true,fontsize=\small,frame=lines, framesep=2mm, tabsize=2,numbersep=5pt]{java}{../../src/android/firstapp/8.java}
\caption{Obtendo extras passados através do \texttt{Intent}}
\end{listing}

Agora para mostrar a mensagem na tela, você precisa criar um \texttt{TextView}\footnote{\href{http://developer.android.com/reference/android/widget/TextView.html}{http://developer.android.com/reference/android/widget/TextView.html}}

\begin{listing}[H]
\inputminted[linenos=true,fontsize=\small,frame=lines, framesep=2mm, tabsize=2,numbersep=5pt]{java}{../../src/android/firstapp/9.java}
\caption{Método \texttt{onCreate()} recebendo um \texttt{Intent} e mostrando a mensagem}
\end{listing}

Agora que o aplicativo está pronto, é necessário testar, caso tenha um smartphone Android você pode conectá-lo no seu computador e rodar diretamente, senão você deverá rodar em um emulador.


Para rodar diretamente no smartphone:
\be
\item Conecte seu smartphone no computador através do cabo USB. Se estiver desenvolvendo no Windows será preciso instalar os drivers USB do seu dispositivo. Se precisar de ajuda para instalar os drivers acesse: \href{http://developer.android.com/tools/extras/oem-usb.html}{OEM USB}\footnote{\href{http://developer.android.com/tools/extras/oem-usb.html}{http://developer.android.com/tools/extras/oem-usb.html}}
\item Ative o modo \emph{USB Debugging} no dispositivo
	\bi
	\item Para Android 3.2 ou mais antigos, a opção deve estar em \textit{Configurações -> Aplicativos	 -> Desenvolvimento}
	\item Para Android 4.0 e 4.1, a opção está em \textit{Configurações -> Opções do desenvolvedor}
	\item Para Andoird 4.2 e mais novos, a opção está escondida por padrão, para mostrar a opção você deve entrar em \textit{Sobre o telefone} e clicar em \textit{Número da versão} 7 vezes, ao retornar para tela anterior deverá aparecer \textit{Opções do desenvolvedor}
	\ei
\ee

\newpage 
Para rodar no emulador:
\be
\item Abra o \textit{SDK Manager} através do Eclipse em: \textit{Window -> Android SDK Manager}
\item Verifique se, para Android 4.2.2 (API 17) ou outro desejado os seguintes pacotes estejam instalados
	\bi
	\item \textit{SDK Platform}
	\item \textit{ARM EABI v7a System Image} ou
	\item \textit{Intel x86 Atom System Image}
	\ei
\item Verifique também se em \textit{Tools}, os pacotes \textit{Android SDK Tools} e \textit{Android SDK Platform-tools} estão instalado
\item Agora é necessário criar um AVD (Android Virtual Device\footnote{\href{http://developer.android.com/tools/devices/index.html}{http://developer.android.com/tools/devices/index.html}}), no Eclipse vá em \textit{Window -> Android Virtual Device Manager} 
\item No AVD Manager clique em \emph{New}
\item Complete as informações do AVD, dê um nome, plataforma, espaço de armazenamento, quantidade de memória RAM
\item Clique \emph{Create AVD}
\item Selecione o novo AVD no \textit{Android Virtual Device Manager} e clique \emph{Start}
\item Quando o emulador terminar de carregar, destrave a tela do emulador
\ee

Agora para rodar o aplicativo basta clicar em \emph{Run} na barra de tarefas do Eclipse e selecionar \textit{Android Application} na janela \emph{Run as}. O Eclipse irá instalar o APK e abrir o aplicativo automaticamente.

\begin{figure}[H]
  \centering
  \includegraphics[width=.4\textwidth]{figuras/6-criando-app.png}
  \caption{Primeira tela do primeiro aplicativo}
  \label{fig:f}
  \includegraphics[width=.4\textwidth]{figuras/7-criando-app.png}
  \caption{Primeira tela após escrever texto na caixa de texto}
  \label{fig:g}
\end{figure}
\begin{figure}[H]
  \centering
  \includegraphics[width=.4\textwidth]{figuras/8-criando-app.png}
  \caption{Segunda tela mostrando a mensagem enviada}
  \label{fig:h}
\end{figure}

%%%%%%% DESIGN %%%%%%%%%%
\cha{design}{Design}

%%% LIST VIEW %%%%
\section{Listas \texttt{(ListView)}}\footnote{Documentação \texttt{ListView}:\href{http://developer.android.com/reference/android/widget/ListView.html}{http://developer.android.com/reference/android/widget/ListView.html}}
Listas são uma das formas mais simples e mais poderosas de se mostrar informações ao usuário de forma simples e objetiva, a \texttt{ListView} é altamente customizável através de adaptadores.

\begin{figure}[H]
  \centering
  \includegraphics[width=.15\textwidth]{figuras/design/listview-scheme.png}
  \caption{Esquema de uma lista}
  \label{fig:e}
\end{figure}

Um item individual da lista pode ser selecionado, essa seleção pode acionar uma outra tela com detalhes do item.


\begin{figure}[H]
  \centering
  \includegraphics[width=.35\textwidth]{figuras/design/listview-scheme2.png}
  \caption{Detalhes de um elemento da lista}
  \label{fig:e}
\end{figure}

\newpage

A construção é simples, você pode começar mudando o layout de \texttt{RelativeLayout} para \texttt{LinearLayout}, a diferença entre eles é que o \texttt{RelativeLayout} do exemplo passado permite você posicionar elementos uns relativos aos outros enquanto que o \texttt{LinearLayout} segue uma estrutura, vertical ou horizontal. Ambas podem ser aninhadas uma dentro de outra.

\begin{listing}[H]
\inputminted[linenos=true,fontsize=\small,frame=lines, framesep=2mm, tabsize=2,numbersep=5pt]{xml}{../../src/android/design/layout_linear.xml}
\caption{Layout \textit{Linear} no \texttt{activity\_main.xml}}
\end{listing}

Agora, você pode arrastar um \texttt{ListView} para o layout ou criar um manualmente com o seguinte código XML

\begin{listing}[H]
\inputminted[linenos=true,fontsize=\small,frame=lines, framesep=2mm, tabsize=2,numbersep=5pt]{xml}{../../src/android/design/listview.xml}
\caption{Código XML de um \texttt{ListView}}
\end{listing}

Agora para popular a lista, você precisa criar um \texttt{string-array} no arquivo \texttt{\textcolor{mygreen}{strings.xml}} com os elementos que deseja colocar na lista.

\begin{listing}[H]
\inputminted[linenos=true,fontsize=\small,frame=lines, framesep=2mm, tabsize=2,numbersep=5pt]{xml}{../../src/android/design/string-array.xml}
\caption{\texttt{string-array} populada com elementos}
\end{listing}

E por final, escrever o código que irá preencher a lista com as \textit{strings} desse \textit{array}.

\begin{listing}[H]
\inputminted[linenos=true,fontsize=\small,frame=lines, framesep=2mm, tabsize=2,numbersep=5pt]{java}{../../src/android/design/listactivity.java}
\caption{Código de uma \texttt{activity} com lista clicável}
\end{listing}

Adaptadores são usados para prover dados para o \texttt{ListView}, ele define como cada linha será mostrada. Um adaptador extende a classe \texttt{BaseAdapter}, o Android provê alguns adaptadores padrões, no caso estamos usando o \texttt{ArrayAdapter} que é para manusear dados em \textit{arrays}.

Dentro da interface \texttt{OnItemClickListener} você pode configurar a ação de clicar em um dos itens da lista, nesse exemplo é criado um \texttt{Toast} que mostra uma mensagem com o texto do item na lista porém, qualquer ação pode ser executada incluindo abrir uma nova \texttt{activity} que esteja relacionada a esse item.

Você pode também herdar da classe \texttt{ListActivity} para uma forma mais simples de manusear \texttt{ListViews}. Você não precisa atribuir um layout a \texttt{ListActivity} contém uma \texttt{ListView} padrão. Caso voce precise colocar mais \texttt{Views} em seu layout, você \emph{deve} colocar a \texttt{ListView} em seu layout com o id \texttt{''@android:id/list''}.

\begin{figure}[H]
  \centering
  \includegraphics[width=.7\textwidth]{figuras/design/lista.png}
  \caption{Lista simples}
  \label{fig:e}
\end{figure}

Obs.: O \texttt{Toast} é esse pequeno retângulo preto com uma mensagem, ele aparece e desaparece rapidamente apenas para mostrar uma mensagem ao usuário.

%%% LIST ADAPTER %%%
\section{Listas Compostas}
É possível compor um item da lista colocando mais elementos no mesmo além de um texto. Para isso você precisa criar um novo arquivo XML que irá definir a customização de cada linha de uma \texttt{ListView}, defina um arquivo chamado \texttt{\textcolor{mygreen}{item.xml}}.

\newpage

\begin{listing}[H]
\inputminted[linenos=true,fontsize=\small,frame=lines, framesep=2mm, tabsize=2,numbersep=5pt]{xml}{../../src/android/design/item.xml}
\caption{Código do arquivo \texttt{\textcolor{mygreen}{item.xml}}}
\end{listing}	

Agora você precisa usar um adaptador para mostrar esse layout customizado em cada linha da lista, uma maneira fácil é usando a classe \texttt{SimpleAdapter}.

\begin{listing}[H]
\inputminted[linenos=true,fontsize=\small,frame=lines, framesep=2mm, tabsize=2,numbersep=5pt]{java}{../../src/android/design/customlist.java}
\caption{Código da lista customizada}
\end{listing}	

\begin{figure}[H]
  \centering
  \includegraphics[width=.475\textwidth]{figuras/design/lista-composta.png}
  \caption{Lista Composta}
  \label{fig:e}
\end{figure}

\section{Listas expandíveis \texttt{(ExpandableListView)}}
\section{Grades \texttt{(GridView)}}
\section{Abas \texttt{(TabView)}}
\section{Arrastar \texttt{(Swipe View)}}
\section{Menu lateral}

\end{singlespace}

\begin{singlespace}

%\lstinputlisting[float, language=java, label=cod:exemplo, caption={Exemplo ... \citep{muchnick}}]{../../src/android/teste.java}

\end{singlespace}


%%%%%%%%%%%%%%%%%%%%%%%%%%%%%%%%%%%%%%%%%%%%%%%%%%%%%%%%%%%%%%%%%%%%%%%%%%%%%%%%

\singlespace
\selectlanguage{brazil}
\cleardoublepage
\thispagestyle{empty}
\phantomsection
\doublespace


\appendix
\addtocontents{toc}{\protect\setcounter{tocdepth}{0}} % as seções do apêndice não aparecem do sumário com este comando...
%%%%%%%%%%%%%%%%%%%%%%%%%%%%%%%%%%%%%%%%%%%%%%%%%%%%%%%%%%%%%%%%%%%%%%%%%%%%%%%%
\ape{apen}{Especificação blá, blá, blá}

Isto é um apêndice...

%%%%%%%%%%%%%%%%%%%%%%%%%%%%%%%%%%%%%%%%%%%%%%%%%%%%%%%%%%%%%%%%%%%%%%%%%%%%%%%%
\addtocontents{toc}{\protect\setcounter{tocdepth}{1}}

\end{document}
